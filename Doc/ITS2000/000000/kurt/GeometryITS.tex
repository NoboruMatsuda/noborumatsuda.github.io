
\section{Geometry Theorem Proving}

\subsection{The domain}

In geometry theorem proving, the operators to be applied are geometry
theorems.  A state is defined by a geometric configuration, the known
geometric relations (the facts, for short) in the configuration, and a
goal(s) to prove.  The subjects of state transition are not only the
facts and the goal, but also the configuration; the configuration is
changed by a construction.

It is well known that, for geometry theorem proving, forward inference
works more efficient than backward inference \cite{Nevins75}.  Indeed,
forward inference suppresses the subgoaling by adding known facts
without blowing up the inference, because there is no infinite chain of
forward inference given each single fact is allowed to be asserted once.
However, it is not easy to realize all the applicable theorems at a
state, because the problem figure might be complicated enough to hinder
the students in reading a necessary configuration off the problem
figure.

Backward inference is not necessarily needed to prove theorems that do
not require a construction of auxiliary lines.  However, it is needed
for the theorems that require a construction.  Through the experiments
with GRAMY, which is introduced in the following section, we have
observed that most of the auxiliary lines and points are constructed by
backward inference.  That is, most constructions are determined by
partially overlapping a configuration of known theorem against the
problem figure to see the lacking segments.  However, it is notoriously
difficult for human beings to retrieve an appropriate theorem from their
memory to conduct a construction by the partial matching method.  So, we
need to help students to see how the partial overlapping takes place.  

Besides the construction, most importantly for human beings, backward
inference helps to think in goal oriented way, and it is more natural
and rational way of reasoning for human beings.  *** ANY CITATION FOR
THIS CLAIM?  *** As we discussed above, this requires a backing-up
procedure which is not easy for students to maintain.  

[[ SCHEINES' WORK GIVES ANY SUPPORT HERE? ]]  \cite{Scheines94}

In sum, we need to teach students forward inference to deduce new facts,
and backward inference to carry out a construction and to maintain a
goal directed reasoning.  The latter strategy needs backing-up at an
impasse.  The main difficulties to teach those topics push us to provide
students with scaffolding on the theorem application, the construction,
and the backing-up procedure.

\subsection{Practical ITSs on Geometry Theorem Proving}

Many ITSs and CAIs have been developed for Geometry Theorem Proving so
far, yet no one has developed an ITS to teach auxiliary line
construction.  

\GeometryTutor, one of the earliest geometry ITSs, developed by Anderson
\textit{et al.}~\cite{Anderson85}, provides students with a GUI where
the students can build up a partial problem tree.  That is, they can
build a proof by making a connection between the givens placed at the
bottom of a screen and the goal placed at the top of the screen.  The
students can use graphical icons representing the geometric theorems as
the connector.  They can either hang an icon from a current goal to
establish new subgoals, or place an icon to hook the known facts on it
to derive a new fact.  The former corresponds to a backward reasoning,
and the latter corresponds to a forward reasoning.  It is expected that
the students can perform bi-directional search naturally as the human
experts do.  Furthermore, \GeometryTutor\ can provide students with
hints on theorem application in terms of the structure of a problem
tree.  However, since it utilized a problem tree as a model of proof, it
could not provide any scaffolding on backing-up, and it has potential
flaws discussed in the previous section.

A recent descendant, \Angle\ \cite{Koedinger93}, extends the tutoring
capabilities of \GeometryTutor\ by providing the students with a
diagrammatic representation of the theorems, called the diagrammatic
configuration schema, instead of the labeled icons used in
\GeometryTutor.  It actually provides students with a better help in the
sense of reifying the theorem applications; students could ``see'' how
to apply theorems.  However, it still uses a partial problem tree in the
interface with which students conduct a proof.

Several ITSs and CAIs have been developed without explicitly using
AND/OR trees.  For example, \Cabri\ \cite{CABRI} provides students with
fully manipulations on the problem configuration.  Students can change
the shape of the configuration while remaining the certain constraints
given in the problem.  This facilitates students to explore the
properties held in the configuration.  By changing a coordinate of the
apex of an isosceles triangle, for example, the students can see that
the apex moves on the perpendicular bisector.  Although that kind of
scaffolding works fairly well, since they do not directly deal with the
problem-solving strategies, they should be placed as the different kind
of instructional tools than the one addressed in this paper.

