
\section{A Reification of Strategy for Geometry Theorem Proving}
\label{sec:gw}

\subsection{GRAMY: A Geometry Theorem Prover for Auxiliary Line Problem}

[[ GIVE A BRIEF INTRODUCTION OF GRAMY HERE ]]


\subsection{Reification of a Search Tree}

To design a GUI that reifies a search tree, we started with showing each
state along with the applicable theorems (i.e., DSs), and allow students
to turn over the states as if they turn the pages each showing a state.
Figure~\ref{fig:bi}~shows the basic idea.  The left hand side of the
figure represents a model of proof in GRAMY.  Meanwhile, the right hand
side of the figure illustrates a screen shot in a geometry learning
environment.  Each window (called \StateView) corresponds to a single
node in the search tree (indicated by a dotted line~\texttt{(b)}).  The
solid window shows the current state and only this window appears on the
screen.  The other windows drawn by the broken lines, corresponding to
ancestor states, are disappeared once turned over.  So, the gray thick
arrow represents chronological progress of a search (indicated by a
dotted line~\texttt{(a)} associating the pile with several node
expansions in the search space specified with thicker links).

\begin{figure}[tb]
 \center
 \resizebox{\textwidth}{!}{\includegraphics{../../Figure/GramyModel}}
 \caption{Reified model of geometry proof}
 \label{fig:bi}
\end{figure}

Figure~\ref{fig:sv}~is an actual view of \StateView\ taken from a
computer screen.  It includes the problem figure, the goal to prove, the
facts deduced so far starting with the given premises, and the
applicable geometry theorems.  By ``applicable,'' we mean either the
theorem has a conclusion that matches against the goal (thus, can be
used for a backward inference), or it has all the premises satisfied
(thus, it immediately derives a new factual statement).  The goal to
prove and the known facts are highlighted on the problem figure with
separate colors.  Observe, in Figure~\ref{fig:sv}, the goal $\angle APQ
= \angle MQB$, and the two facts $AP = QB$ and $AM = AB$ are all
highlighted.  When one of the statements (i.e., either the goal or the
facts) is clicked, corresponding geometric objects are blinking in the
problem figure.

The applicable DSs are lined up with the iconic representation
illustrating configuration of corresponding theorem (a dotted line
\texttt{(d)} in Figure~\ref{fig:bi}).  The experiment of GRAMY running
on a corpus of non-auxiliary line problems from the literature showed
that the average branching factor is less than 4.  It supports our
design policy of using the iconic buttons.  Furthermore, it has also
been observed that a number of possible constructions with a particular
DS runs up to 40.  Thus, instead of lining up all the possible
constructions in \StateView, only the icons of DSs used to determine the
construction are appeared at the bottom of \StateView.  When those icons
are clicked, all the constructions determined by the corresponding DS
appear in a pop-up window.  Figure~\ref{fig:popup}~ shows an example of
the pop-up window showing possible constructions by the theorem of
congruent triangles.

\begin{figure}[tb]
 \center
 \resizebox{\textwidth}{!}{\includegraphics{../../Image/P116-StateView0+}}
 \caption{A representation of a proof step.}
 \label{fig:sv}
\end{figure}

\begin{figure}[tb]
 \center
 \resizebox{.8\textwidth}{!}{\includegraphics{../../Image/P116-PopupDsIcon-0-TriCong}}
 \caption{An example of DSs in a pop-up window.}
 \label{fig:popup}
\end{figure}

A pop-up window is also used for DSs applied by forward inference.
Since GRAMY applies, in forward inferences, all the applicable DSs at
once, there might be a number of DSs in those steps.  Then, an existence
of a forward inference at a state is shown with an icon saying
``\texttt{FORWARD}.''  (See Figure~\ref{fig:sv}.)

The third kind of icons are for backward inference.  They are lined one
by one at the bottom of \StateView.

When an icon in a pop-up window or at the bottom of \StateView\ is
clicked, a configuration of the corresponding theorem is overlapped
against the problem figure, and the theorem is appeared in a separate
window, called diagrammatic schema browser (or \DsBrowser\ for short),
with which students can browse all the geometry theorems.  (See
Figure~\ref{fig:dsb}.)  \DsBrowser\ shows the sentential expression of a
theorem together with its configuration.  In Figure~\ref{fig:ov}, an
application of the theorem of triangle congruent illustlates the
auxiliary lines overlapping against the problem figure in \StateView.

\begin{figure}[tb]
 \center
 \resizebox{\textwidth}{!}{\includegraphics{../../Image/P116-PopupDsIcon-0-TriCong-overlap}}
 \caption{An example of reified DS application.}
 \label{fig:ov}
\end{figure}

In sum, \StateView\ has potential abilities to offer students with the
following scaffolding per se.

\begin{itemize}
 \item [(S1)] Each state in a search tree is reified by visualizing a
       problem figure, goal to be proven, known facts, and applicable
       theorems (i.e., DSs).
 \item [(S2)] The successors expanded at each node in the search tree
       are reified as the clickable icons.  
 \item [(S3)] An application of a theorem is reified by overlapping a
       configuration against the problem figure, and also showing
       sentential expressions in \DsBrowser.  
 \item [(S4)] Specific to the auxiliary line problems, but construction
       is reified as an overlapped configuration against the problem
       figure.  
\end{itemize}

\begin{figure}[tb]
 \center
 \resizebox{\textwidth}{!}{\includegraphics{../../Image/BsBrowser-Ftheory}}
 \caption{Diagrammatic Schema Browser}
 \label{fig:dsb}
\end{figure}

Finally, when an icon is ``selected'' by having a double click on it,
then the selected icon is highlighted, and a new state is expanded by
applying the selected DS to the current state (i.e., where the icon is
clicked).  Then, the contents of \StateView\ are updated accordingly.
The figure is modified if the applied theorem is for a construction.
The goal is replaced if the theorem is applied in backward manner.  The
new facts are added if the icon for forward inference is selected.  

Now, the students can observe how a proof unfolds in \StateView\ by
keeping to select an applicable DS (i.e., an icon) at each state.  They
can return to a previous state by clicking the back button on
\StateView\ (see \fbox{\texttt{Next}} and \fbox{\texttt{Previous}} buttons
shown in Figure~\ref{fig:sv}).  The number at the right-bottom corner of
\StateView\ shows a depth of the search.  This way, the students can
freely explore through a search tree back and forth.

Most importantly, the students are allowed to conduct a depth-limited
search; that is, they go down a single path in the search tree until
they hit a cutoff\footnote{The cutoff is determined according to the
depth of a proof found by GRAMY invoking the iterative-deepening
search.} or a dead-end.  When they reach a cutoff or a dead-end, no
applicable DS appears so they know that they are at an impasse.  At that
point, they can ``back-up'' to a previous state where a selected DS is
highlighted, and thus they can take another path.

In sum, this state exploration facility provides students with powerful
scaffolding on learning the problem-solving strategies as follows;

\begin{itemize}
 \item [(S5)] State expansion is reified as a ``selection'' of an
       applicable DS.  
 \item [(S6)] Backing-up is reified as turning over the states in
       backward way, and selecting an alternative icon.
\end{itemize}

