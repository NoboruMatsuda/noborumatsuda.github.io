This paper addresses the design issues of learning environment for
problem solving in a complex domain, such as geometry theorem proving
with construction.  We first discuss the difficulties to study
problem-solving strategies (i.e., forward and backward inference, and
backing-up at impasse) in such a domain, and examine the three kinds of
representations to describe a problem solving with search; viz., a
solution tree, a search tree, and a problem tree.  We then claim that it
is a search tree that we need to reify to teach students the
problem-solving strategies.  Finally, we show a geometry learning
environment with a GUI that reifies a search to prove geometry theorems
with auxiliary line constructions.  Several powerful scaffolding
techniques in this learning environment are discussed followed by a
conclusion that our reification technology might have a potential
ability to facilitate students to learn desired problem-solving
strategies.

